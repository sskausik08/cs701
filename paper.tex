\documentclass[]{article}

\usepackage{hyperref}

%opening
\title{Automatic extraction of network signatures from malware code}
\author{Arjun Gurumurthy, Kausik Subramanian}

\begin{document}

\maketitle

\begin{abstract}

\end{abstract}

\section{Motivation}
Malware, short for malicious software, 
is one of the primary threats to modern-day networks, especially
with the mass proliferation of the Internet. These are used to 
disrupt computer operations, extract sensitive information, and 
gain access to private systems.  Infected hosts can participate 
in distributed denial-of-service attacks (DDoS), these class of 
attacks are very hard to detect and mitigate, leading to immense
loss of revenues. A new shift is towards creating malware for 
profit called botnets, in which a customer can buy a network of
infected hosts for a price. In such a setting, the malware is
specifically designed to evade detection, thus motivating the need
for better detection tools for malware. 

The most prevalent form of malware detection is endhost-based detection
(commonly known as anti-virus software). Detected malware's code is hashed
and the code signatures are uploaded to a central repository. By maintaining
an up-to-date repository of malware signatures, the endhost software can
detect malware by comparing program signatures with the repository. However, 
endhost-based malware detection can fail due to various reasons~\cite{networksig}:
\begin{itemize}
	\item \textbf{Anti-virus software is not operational or up to date.} 
	Due to this, the endhost machine can be infected by malware without any
	noticeable effects, thus the user will be oblivious of the infection.
	\item \textbf{Packer programs.} To render programs stealthy, malware authors
	employ packer programs~\cite{packer}. Packers change the program content so that
	its signature differs but its functionality has not changed, thus avoiding
	detection by endhost signature-based software.
\end{itemize}

Instead, a intrusion-detection mechanism integrated 
directly into the service provider’s network offers a
much-needed additional layer of protection. While the malware
code is different, the protocol of the malware used to talk to its
control server (which issues orders to the malware) will not vary. 
Thus, by generating accurate \emph{network signature} of the malware (which
can informally defined as the set of messages sent and received by the malware to
initiate an action), we can use the intrusion detection system in the
network (for e.g., Bro~\cite{bro}) to search for these particular signatures,
and can be used to infer the presence of the malware in the network. However,
manually inspecting malware code is an intractable approach (signatures may 
need to be extracted in real-time). Thus, we propose \emph{automatic extraction 
of accurate network signatures from malware code using different static
analysis techniques.} 

\section{Challenges}
We abstract the problem to the following: given a program point,
can we automatically provide an ordered set of symbolic packets 
that must be received (and sent) by the client such that the program 
reaches the program point. The symbolic packets can be used as
network signatures and used to detect malware (a set of packets
satisfying the symbolic packets in order will infer that there 
is a malware with high confidence). By constructing symbolic packet
signatures, we can generalise the intrusion policy and reduce 
true negatives. 

\bibliographystyle{abbrv}
\bibliography{references}

\end{document}
